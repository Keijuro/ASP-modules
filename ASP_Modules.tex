% This is "aamas2012 .tex" August 2012 
% This file should be compiled with "aamas2012 .cls" 
% This example file demonstrates the use of the 'aamas2012 .cls'
% LaTeX2e document class file. It is for those submitting
% articles to AAMAS 2012  conference. This file is based on
% the sig-alternate.tex example file.
% The 'sig-alternate.cls' file of ACM will produce a similar-looking,
% albeit, 'tighter' paper resulting in, invariably, fewer pages.
% than the original style ACM style.
%
% ----------------------------------------------------------------------------------------------------------------
% This .tex file (and associated .cls ) produces:
%       1) The Permission Statement
%       2) The Conference (location) Info information
%       3) The Copyright Line with AAMAS data
%       4) NO page numbers
%
% as against the acm_proc_article-sp.cls file which
% DOES NOT produce 1) thru' 3) above.
%
% Using 'aamas2012 .cls' you don't have control
% from within the source .tex file, over both the CopyrightYear
% (defaulted to 200X) and the IFAAMAS Copyright Data
% (defaulted to X-XXXXX-XX-X/XX/XX).
% These information will be overwritten by fixed AAMAS 2012  information
% in the style files - it is NOT as you are used with ACM style files.
%
% ---------------------------------------------------------------------------------------------------------------
% This .tex source is an example which *does* use
% the .bib file (from which the .bbl file % is produced).
% REMEMBER HOWEVER: After having produced the .bbl file,
% and prior to final submission, you *NEED* to 'insert'
% your .bbl file into your source .tex file so as to provide
% ONE 'self-contained' source file.
%

\newtheorem{note}{Note}
\newtheorem{example}{Example}
\newtheorem{definition}{Definition}

% This is the document class for full camera ready papers and extended abstracts repsectively 

\documentclass{aamas2012}

% if you are using PDF LaTex and you cannot find a way for producing
% letter, the following explicit settings may help
 
\pdfpagewidth=8.5truein
\pdfpageheight=11truein

\begin{document}

% In the original styles from ACM, you would have needed to
% add meta-info here. This is not necessary for AAMAS 2012  as
% the complete copyright information is generated by the cls-files.


\title{Representing Agent reasoning with Meta-Knowledge on ASP Modules Combination}

% AUTHORS


% For initial submission, do not give author names, but the
% tracking number, instead, as the review process is blind.

% You need the command \numberofauthors to handle the 'placement
% and alignment' of the authors beneath the title.
%
% For aesthetic reasons, we recommend 'three authors at a time'
% i.e. three 'name/affiliation blocks' be placed beneath the title.
%
% NOTE: You are NOT restricted in how many 'rows' of
% "name/affiliations" may appear. We just ask that you restrict
% the number of 'columns' to three.
%
% Because of the available 'opening page real-estate'
% we ask you to refrain from putting more than six authors
% (two rows with three columns) beneath the article title.
% More than six makes the first-page appear very cluttered indeed.
%
% Use the \alignauthor commands to handle the names
% and affiliations for an 'aesthetic maximum' of six authors.
% Add names, affiliations, addresses for
% the seventh etc. author(s) as the argument for the
% \additionalauthors command.
% These 'additional authors' will be output/set for you
% without further effort on your part as the last section in
% the body of your article BEFORE References or any Appendices.

%\numberofauthors{8} %  in this sample file, there are a *total*
% of EIGHT authors. SIX appear on the 'first-page' (for formatting
% reasons) and the remaining two appear in the \additionalauthors section.
%

\numberofauthors{3}

\author{
% You can go ahead and credit any number of authors here,
% e.g. one 'row of three' or two rows (consisting of one row of three
% and a second row of one, two or three).
%
% The command \alignauthor (no curly braces needed) should
% precede each author name, affiliation/snail-mail address and
% e-mail address. Additionally, tag each line of
% affiliation/address with \affaddr, and tag the
% e-mail address with \email.
% 1st. author
\alignauthor
Tony Ribeiro\\
       \affaddr{National Institute of Informatics}\\
       \affaddr{Tokyo, Japan}\\
       \email{ribeiro@nii.ac.jp}
% 2nd. author
\alignauthor
Katsumi Inoue\\
       \affaddr{National Institute of Informatics}\\
       \affaddr{Tokyo, Japan}\\
       \email{ki@nii.ac.jp}
% 3rd. author
\alignauthor
Gauvain Bourgne\\
       \affaddr{????}\\
       \affaddr{Paris, France}\\
       \email{bourgne@nii.ac.jp}
}

%\and  % use '\and' if you need 'another row' of author names

% 4th. author
%\alignauthor Lawrence P. Leipuner\\
%       \affaddr{Brookhaven Laboratories}\\
%       \affaddr{Brookhaven National Lab}\\
%       \affaddr{P.O. Box 5000}\\
%       \email{lleipuner@researchlabs.org}

% 5th. author
%\alignauthor Sean Fogarty\\
%       \affaddr{NASA Ames Research Center}\\
%       \affaddr{Moffett Field}\\
%       \affaddr{California 94035}\\
%       \email{fogartys@amesres.org}

% 6th. author
%\alignauthor Charles Palmer\\
%       \affaddr{Palmer Research Laboratories}\\
%      \affaddr{8600 Datapoint Drive}\\
%       \affaddr{San Antonio, Texas 78229}\\
%       \email{cpalmer@prl.com}

%\and

%% 7th. author
%\alignauthor Lawrence P. Leipuner\\
%       \affaddr{Brookhaven Laboratories}\\
%       \affaddr{Brookhaven National Lab}\\
%       \affaddr{P.O. Box 5000}\\
%       \email{lleipuner@researchlabs.org}

%% 8th. author
%\alignauthor Sean Fogarty\\
%       \affaddr{NASA Ames Research Center}\\
%       \affaddr{Moffett Field}\\
%       \affaddr{California 94035}\\
%       \email{fogartys@amesres.org}

%% 9th. author
%\alignauthor Charles Palmer\\
%       \affaddr{Palmer Research Laboratories}\\
%       \affaddr{8600 Datapoint Drive}\\
%       \affaddr{San Antonio, Texas 78229}\\
%       \email{cpalmer@prl.com}

%}

%% There's nothing stopping you putting the seventh, eighth, etc.
%% author on the opening page (as the 'third row') but we ask,
%% for aesthetic reasons that you place these 'additional authors'
%% in the \additional authors block, viz.
%\additionalauthors{Additional authors: John Smith (The Th{\o}rv{\"a}ld Group,
%email: {\texttt{jsmith@affiliation.org}}) and Julius P.~Kumquat
%(The Kumquat Consortium, email: {\texttt{jpkumquat@consortium.net}}).}
%\date{30 July 1999}
%% Just remember to make sure that the TOTAL number of authors
%% is the number that will appear on the first page PLUS the
%% number that will appear in the \additionalauthors section.

\maketitle

\begin{abstract}
	In this work, we focus on multi-agent systems in dynamic environment.
	Our interest is about individual agent reasoning in such environment.
	For reasoning in dynamic environment, an agent needs to be able to manage his knowledge in a non-monotonic way.
	To reach his goals in a changing environment, an agent needs to adapt his behaviours regarding the current state of the world.
	Our objective is to define a method which makes easier to design agent knowledge and reasoning in such environment.
	We use the expressivity of answer set programming to represent agent knowledge.
	To design agent reasoning, we propose a method based on ASP modules combination and meta-knowledge.
	We also propose a framework to implement and use this method in multi-agent systems.
\end{abstract}

% Note that the category section should be completed after reference to the ACM Computing Classification Scheme available at
% http://www.acm.org/about/class/1998/.

\category{H.4}{Information Systems Applications}{Miscellaneous}

%A category including the fourth, optional field follows...
%\category{D.2.8}{Software Engineering}{Metrics}[complexity measures, performance measures]

%General terms should be selected from the following 16 terms: Algorithms, Management, Measurement, Documentation, Performance, Design, Economics, Reliability, Experimentation, Security, Human Factors, Standardization, Languages, Theory, Legal Aspects, Verification.

\terms{Design}

%Keywords are your own choice of terms you would like the paper to be indexed by.

\keywords{Multi-Agents System, Answer Set Programming, Meta-knowledge, ASP modules}

\section{Introduction}

\section{State of the art}

\section{Dynamic environment}

	Our interest is about representing agent reasoning in dynamic environment.
	To make our work more understandable we will follow an intuitive example along our propositions: a survival game which represent a MAS in a dynamic environment.
	In this game there are three groups of agents: wolfs, rabbits and flowers.
	Each kind of agent have specific goals and behaviours.
	To be simple, wolfs eat rabbits and rabbits eat flowers.
	
	Wolfs have only one goal: feed themselves.
	To reach this goal they have to catch and eat rabbits.
	A wolf can be in two situations: a prey is in sight or not.
	If there is no rabbit in the sight range of a wolf, the predator have to explore his environment to find one.
	When a prey is spotted a wolf will try to perform a sneaky approach if he is not spotted himself, otherwise our predator will rush on his target.
	To resume, a wolf have three behaviours: exploration, approach and attack.

\section{ASP modules}

	An ASP module is an ASP program which have a specific form and a specific use.
	The first advantage of these modules is their simplicity: a module is a little program which represent specific knowledge.
	We can have a module which contain observations about surroundings,
	an other one to define what is a prey and a module dedicated to compute path.
	To obtain all paths to surroundings preys an agent will combines this three modules.
	By combining modules an agent can produce knowledge, it the purpose of our ASP modules.

\subsection{Background theory}

	\begin{definition}[Rules module]
		A rules module is a set of rules which represent knowledge about a specific domain.
		The content of such module is static: it does not change regarding time.
		The purpose of these modules is to organise knowledge representation and produce
		new knowledge by combine it with others modules.
	\end{definition}

\subsection{Observations}

	\begin{definition}[Observations module]
		An observations module is a set of facts which represent related observations.
		The content of such module is dynamic: it change regarding time.
		An agent use it like a specific memory database.
		The purpose of these modules is to organise observations to facilitate their use and update.
	\end{definition}

\subsection{Meta-knowledge}

	\begin{definition}[Meta-knowledge module]
		A meta-knowledge module is a set of rules which define the conditions to use an ASP module.
		The content of such module does not change regarding time.
		It contains knowledge on modules combination.	
		The purpose of these modules is to guide reasoning and represent dynamic behaviours.
	\end{definition}

\section{Experiments}

\section{Conclusions}

	

%
% The following two commands are all you need in the
% initial runs of your .tex file to
% produce the bibliography for the citations in your paper.
\bibliographystyle{abbrv}
\bibliography{ASP_Modules}  % sigproc.bib is the name of the Bibliography in this case
% You must have a proper ".bib" file
%  and remember to run:
% latex bibtex latex latex
% to resolve all references
%
% ACM needs 'a single self-contained file'!
%

\nocite{*}

\end{document}
